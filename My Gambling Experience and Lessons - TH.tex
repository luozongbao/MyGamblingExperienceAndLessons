%!TEX program = xelatex
%!TEX encoding = UTF-8 Unicode 
\documentclass[12pt]{article}
\usepackage{fontspec}
\defaultfontfeatures{Mapping=tex-text}
\usepackage{xunicode}
\usepackage{xltxtra}
\setmainfont{TH SarabunPSK}
\XeTeXlinebreaklocale 'th_TH' 
\usepackage{amsmath}
\usepackage{amsfonts}
\usepackage{amssymb}
\usepackage{csquotes}
\usepackage{mathtools}
\usepackage{pgfplots}

\usepackage{color}
\definecolor{white}{rgb}{1,1,1}
\definecolor{dkgreen}{rgb}{0,0.6,0}
\definecolor{gray}{rgb}{0.5,0.5,0.5}
\definecolor{mauve}{rgb}{0.58,0,0.82}
\definecolor{cream}{rgb}{1, 0.992, 0.816}
\definecolor{lightyellow}{rgb}{1, 1, 0.875}


\author{อธิปัตย์ ล้อวงศ์งาม}
\title{\scalebox{2}{\textbf{ประสบการณ์การพนันและการลงทุน}}}
\date{\today}

\begin{document}
\pagecolor{lightyellow}
\maketitle
\newpage
\tableofcontents

\newpage
\section{คำนำ}
ผมได้เริ่มเข้าสู่วงการการเล่นการพนันตั้งแต่ปี 2555 เมื่อผมได้เลื่อนขั้นเป็นผู้บริหารในบริษัทแห่งหนึ่งและทริปท่องเที่ยวแรกของผู้บริหารได้จัดไปคือเรือสำราญแห่งหนึ่งในฮ่องกง มันเป็นประสบการณ์ครั้งแรกของผมในคาสิโนจริง ๆ ที่ได้เล่นที่แม้แต่เพียงเล่นเพื่อทดลองบันเทิง แต่ด้วยความที่ผมมีความเป็นนักเล่นเกมส์ไม่ยอมแพ้ และคิดว่าเรื่องเหล่านี้เป็นคณิตศาสตร์ วิทยาศาสตร์ หัววิศวกรอย่างผมต้องสามารถคำนวณหาผลลัพธ์หรือ Pattern ได้สิน่า และนั่นเป็นจุดเริ่มต้นของการเดินทางอันยาวนานครั้งนี้ของผม

บ้านผมเป็นครอบครัวคนจีนที่มองเห็นการพนัน การลุงทุนเป็นสิ่งเดียวกันเป็นปีศาจที่จะทำให้ทุกคนที่เข้าไปยุ่งเกี่ยวข้องและคนรอบ ๆ ตัวของเขาต้องเดือนร้อนวุ่นวาย จนมีคำว่า

\begin{displayquote}
โจรขึ้นบ้านร้อยครั้งไม่เท่าไฟไหม้บ้าน 1 ครั้ง ไฟไหม้บ้านร้อยครั้งไม่เท่าติดยาเสพติด 1 ครั้ง ติดยาเสพติดร้อยครั้งยังไม่เท่าติดการพนัน 1 ครั้ง
\end{displayquote}

จากคำกล่าวข้างต้นโดยรวมแล้วปีศาจของปีศาจคือการพนัน และการกล่าวเช่นนั้นก็ไม่ได้ห่างจากความเป็นจริงเลย แต่เราเคยถามกันไหมว่า การพนันแท้จริงแล้วคืออะไรเมื่อเราทั้งหลายต่างก็รู้ว่าในชีวิตคนเรามีความเสี่ยงอยู่ในทุกที่ทุกเวลาทุกโอกาส และในโอกาสต่าง ๆ ทีผลตอบแทนสูงย่อมมีความเสี่ยงที่สูงด้วย แล้วทำไมการพนันมันถึงเป็นปีศาจขนาดนั้น และตกลงว่าอย่างไหนคือการพนันอย่างไหนคือการลงทุน และมีบ้างไหมการพนันแบบไหนที่ควรจะพนัน

เป้าหมายของบทความนี้ไม่ใช่เพื่อการทำให้คนอยากเล่นการพนันมากขึ้นหรือสนใจมาเล่นให้มากขึ้น แต่่อยากให้ความรู้ความเข้าใจที่แท้จริงเกี่ยวกับการพนัน การลงทุนอย่างเป็นระบบ มีทฤษฏีทางคณิตศาสตร์และวิทยาศาตร์ที่เชื่อถือได้มารองรับ รวมทั้งตอบคำถามทั้งหลายของนักพนันให้เห็นว่าสิ่งที่เขาทั้งหลายเชื่อกันว่านำโชค นำผลตอบแทนจากการพนันแท้จริงแล้วเป็นอย่างไร น่าเชื่อต่อไปไหม รวมถึงเรามาดูในเชิงจิตวิทยาของการพนันและการลงทุนว่า ทำไมมันถึงหอมหวานและเลิกได้ยาก มากจนถึงขั้นที่บางคนทำร้ายตัวเอง ขายครอบครัว ขายกิจการ หรือแม้กระทั่งอวัยวะของตัวเองเพื่อที่จะเล่นการพนัน และท้ายที่สุดก็ตัดสินใจปลิดชีพตัวเองเมื่อมีภาระหนี้สินเกินไปกว่าที่จะถอนได้

\newpage
\section{กลยุทธ์การเดิมพันแบบมือใหม่}
เรามักคิดว่าเกมส์ทั้งหลายที่มีลักษณะมีเพียง 2 ผลลัพธ์ ว่าเป็นเกมส์ประเภท 50-50 แต่ความเป็นจริงแล้วแทบจะไม่มีเกมส์ใดเลยที่จะมีผลลัพธ์เป็น 50-50 จริง ๆ นอกเสียจากเกมส์ที่จำลองขึ้นจาก Computer ซึ่งแม้แต่การโยนเหรียญ "หัวก้อย" ก็ยังมีผลลัพธ์ที่ไม่ใช่ 50-50 โดยที่เหรียญแต่ละประเภทจะมีโอกาสออกหัวออกก้อยแตกต่างกันไปตามประเภทของเหรียญ เทคนิค วิธีการ น้ำหนักในการโยนเหรียญล้วนมีผลต่อการโยนทั้งสิ้น

การเล่นการพนันหนึ่งในคาสิโน (Casino) นั้นมีเกมส์หนึ่งเรียกว่า Baccarat เป็นเกมส์ที่มีวิธีการเล่นคล้าย ๆ กันกับป๊อกเด้งของคนไทยเราแต่สิ่งที่ต่างกันไปก็คือเป็นเกมส์ที่ผู้เล่นจะไม่ได้เล่นมือของตัวเอง ผู้เล่นมีหน้าที่เพียงเลือกฝ่ายที่เรียกว่า Player (สีน้ำเงิน) หรือ Banker (สีแดง) เท่านั้นว่าฝ่ายไหนจะได้คะแนนสูงกว่าก็เป็นฝ่ายชนะไป ดูเผิน ๆ เหมือนจะเป็นเกมส์ 50-50 แต่ความเป็นจริงเป็นเกมส์ที่ค่อนไปทางสีแดง (Banker) จะชนะบ่อยกว่าเล็กน้อย เอาเป็นว่าในบทนี้เราถือซะว่าเป็น 50-50 หรือเรียกเป็นภาษาคณิตศาสตร์ว่า 50\% ไปก่อนละกัน

\subsection{ระบบมาร์ติงเกลส์ Martingale's Strategy}
ผมเคยคิดว่าเกมส์อะไรก็ตามไม่ต้อง 50-50 เป๊ะหรอก เอาแค่ใกล้เคียงก็ถือว่าใช้กลยุทธ์นี้ได้ กลยุทธ์นี้มีชื่อว่า "มาร์ติงเกล" (Martingales) วิธีการใช้กลยุทธนี้ไม่ยาก กลยุทธ์นี้มาจากพื้นฐานว่า "เกมส์มัน 50-50 ยังไงคุณก้ไม่แพ้ตลอดหรอก" ก็แค่คุณลงเงินเท่าตัวของเงินที่คุณเสียก่อนหน้าไปเรื่อย ๆ เพราะมาลองคิดดูว่าโอกาสที่จะแพ้ติด ๆ กัน 10 ครั้งประมาณหนึ่งในพันเท่านั้นเอง หากเรามาคำนวณทางคณิตศาสตร์จะได้ว่า

สมมติว่า $p_l$ เป็นความน่าจะเป็นที่เราจะแพ้ในเกมส์นี้ย่อมเท่ากับ 50\% หรือ $0.5$ หรือเขียนในรูปเศษส่วนได้ว่า $\frac{1}{2}$ หากเราต้องการหาความน่าจะเป็นของการแพ้ติด ๆ กัน $n$ ครั้งเราก็เพียงยกกำลัง $n$ เข้าไปจะได้ว่า เมื่อ $p_{l_n}$ เป็นโอกาสที่เราจะแพ้ติด ๆ กัน $n$ ครั้ง
\begin{center}
$p_{l_n} = (\frac{1}{2})^n$ \\
(2.1)
\end{center}
จากสมการข้างต้นเราจะได้ว่า ที่ $n$ มีค่าต่าง ๆ เช่น 6, 7, 8 โอกาสที่จะแพ้ติด ๆ กัน 6 ครั้ง 7 ครั้ง หรือ 8 ครั้ง จะเท่ากับ
\begin{center}
$p_{l_6}=(\frac{1}{2})^6=\frac{1}{64}=0.015625\approx1.56\%$\\
$p_{l_7}=(\frac{1}{2})^7=\frac{1}{64}=0.0078125\approx0.78\%$\\
$p_{l_8}=(\frac{1}{2})^8=\frac{1}{256}=0.00390625\approx0.4\%$\\ 
\end{center}
เอาล่ะชักตื่นเต้นขึ้นมาแล้ว อย่าเพิ่งรีบร้อนไปเล่น... ถ้ามันง่ายอย่างนั้นคาสิโนคงเจ๊งกันหมดแล้ว... ใจเย็นแล้วนะ... มาเรามาดูกันต่อ... 

แล้วถ้าเรามองในมุมมองที่ว่าหากเรามองในมุมที่เราจะชนะในเกมส์เหล่านี้ล่ะจะมีโอกาสชนะสักเท่าไหร่กันนะ? วิธีการคือ "ผลรวมของความน่าจะเป็นต้องเท่ากับ 1 เสมอ" ดังนั้นถ้าเราสมมติ $p_{w_n}$ คือโอกาสที่เราจะชนะหากเราเล่น Double เงินลงทุนของเราซ้ำ ๆ ไปเรื่อย ๆ เราจะได้ว่า
\begin{center}
$p_{w_n} = 1-p_{l_n}$ \\
$p_{w_n} = 1-(\frac{1}{2})^n$ \\
(2.2)
\end{center}
ดังนั้นโอกาสที่เราจะเล่นชนะในเกมส์เหล่านี้คือ 
\begin{center}
	$p_{w_6}=1-\frac{1}{2}^6=1-\frac{1}{64}=0.984375\approx98.44\%$\\
	$p_{w_7}=1-\frac{1}{2}^7=1-\frac{1}{128}=0.9921875\approx99.22\%$\\
	$p_{w_8}=1-\frac{1}{2}^8=1-\frac{1}{256}=0.99609375\approx99.61\%$\\
\end{center}
เราจะเห็นว่าโอกาสที่เราจะชนะในเกมส์ที่เรา double เงินลงทุนของเราไป 6 ครั้ง 7 ครั้ง หรือ 8 ครั้งจะประมาณ 98.44\%, 99.22\%, 99.61\% ตามลำดับ 

โอกาสที่ชนะเกือบ 100\% โอกาสที่จะแพ้แทบเท่ากับ 0\% แบบนี้มือใหม่หรือนักคณิตศาสตร์มือสมัครเล่นบวกกับความโลภของคนเพียงเล็กน้อย ก็หลงกลได้ง่าย ๆ ผมเองก็เคยหน้ามืดตามัวไปด้วยตัวเลขเหล่านี้ เอาล่ะคราวนี้เรามาดูกันว่าความเป็นจริงแล้วมันมีปัญหาอะไรกับเรื่องนี้

ระบบมาร์ติงเกลกล่าวว่า "เรา Double จำนวนเงินลงทุนไปเรื่อย ๆ จนกว่าจะชนะ แล้วเริ่มนับ 1 ใหม่" ถ้าอย่างนั้นมาดูกันว่าเราต้องลงทุนเท่าไรเพื่อให้ในการเล่นเกมส์ที่เราจะตัดสินที่การแพ้ติด ๆ กัน 6 ครั้ง 7 ครั้งหรือ 8 ครั้ง 

หากว่าการเล่นครั้งแรกเท่ากับ 1 บาทเล่นครั้งที่ 2 ก็เท่ากับ 2 บาทหากครั้งนี้แพ้อีก คราวต่อไปก็ต้องลง 4 บาทไปเรื่อย ๆ จากหลักการนี้เราจะได้ว่าเราต้องลงเงินเป็นอัตตรา $2^{(n-1)}$ ของเงินจำนวนตั้งต้น สมมติว่าจำนวนเงินขั้นต่ำแรกเป็น $a_{min}$ เราจะได้ว่าจำนวนเงินที่เราต้องลงในแต่ละเกมส์ $b_n$ จะเท่ากับ\\
\begin{center}
$b_n=a_{min}*2^{(n-1)}$\\
(2.3)
\end{center}
เพื่อจะหาผลรวมของเงินทุนที่เพียงพอสำหรับเล่นเกมส์ที่สามารถแพ้ได้ติด ๆ กัน $n$ ครั้ง นั่นหมายถึงว่าในตาที่ เราจะต้องชนะในเกมส์ที่ $n$ และผลรวมที่เราต้องคำนวณคือเงินลงทุนตั้งแต่ $n=1$ จนถึง $n-1$ ตามตัวอย่างเบื้องต้นจะได้ว่า
\begin{center}
$b_{total_{n-1}}=a_{min}(b_1+b_2+b_3+...b_{n-1})$\\
$b_{total_{n-1}} = a_{min} * \sum^{n-1}_{i=1}b_i$\\
$b_{total_{n-1}} = a_{min} * (2^{(n-1)}-1)$\\
(2.4)
\end{center}
จาก (2.4) และ (2.3) เราจะเห็นได้ว่าถ้าเราจะชนะในเกมส์ที่ $n$ ใด ๆ เราจะได้เงินจากการชนะครั้งนั้นเป็นจำนวนเท่ากับ $w_n$ เมื่อ $w_n$ เป็นจำนวนเงินรวมที่ชนะในตาที่ $n$ จะได้ว่า
\begin{center}
$w_n=b_n - b_{total_{n-1}}$\\
$w_n=a_{min}*2^{(n-1)} - a_{min}(2^{(n-1)}-1)$\\
$w_n=a_{min}*(2^{(n-1)}-(2^{(n-1)}-1))$\\
$w_n=a_{min}*(2^{(n-1)}-2^{(n-1)}+1)$\\
$w_n=a_{min}*1$\\
(2.5)
\end{center}
เมื่อ $a_{min}$ คือจำนวนเงินที่เล่นตาแรก เราจะได้ว่าในทุก ๆ ครั้งที่เราชนะ 1 ครั้ง ที่ตา $n$ ใด ๆ เมื่อหักลบจำนวนที่เราเสียก่อนหน้าใด ๆ ไปแล้วเราจะได้กำไร $a_{min}$ เสมอ 

แต่ถ้าแพ้ตลอด $n$ ครั้งละเราจะต้องเสียเท่าไหร่
\begin{center}
$b_{total_n}=a_{min}(b_1+b_2+b_3+...b_n)$\\
$b_{total_n-} = a_{min} * \sum^n_{i=1}b_i$\\
$b_{total_n} = a_{min} * (2^n-1)$\\
(2.6)
\end{center}

จาก (2.6) เราจะได้ว่าโดยสรุปแล้ว เราต้องเสี่ยงเงิน $2^n-1$ เพื่อให้ได้เงิน $1$ นั่นหมายถึงว่าในเกมส์ที่คุณพร้อมที่จะแพ้ติด ๆ กัน 6 ตา 7 ตา 8 ตานั้นคุณจะต้องลงเงินจำนวน 63, 127, 255 บาทตามลำดับเพื่อแลกกับชัยชนะเพียง 1 บาท และในขณะที่เงินลงทุนที่เพิ่มขึ้นเป็นทวีคูณนั้นจะเพิ่มขึ้นอย่างรวดเร็ว เช่นในตาที่ 10 คุณอาจต้องเตรียมใจลงเงินถึง 512 บาทเพื่อชนะเพียง 1 บาทและหากตานั้นคุณแพ้อีกรวมทั้งสิ้นคุณต้องเสียเงินทั้งสิ้น 1023 บาทีเดียว
\begin{center}
\begin{tikzpicture}
  \begin{axis}[ 
  	xmin=-1,
  	xmax=10,
    xlabel=$n$,
    ylabel={$f(n) ={2^n}-1$}
  ] 
    \addplot [domain=-1:10]{2^x-1}; 
  \end{axis}
\end{tikzpicture}
\end{center}
นี้คงเป็นคำกล่าวที่เรียกว่า "สายป่านยาว" ในการลงทุนหรือการพนันก็ตามคนที่มีสายป่านย่อมมีโอกาสที่มากกว่า นี้ก็ถูกเป็นเบื้องต้น แต่ที่สำคัญคือเท่าไหร่ถึงจะพอถึงจะเหมาะถึงจะเพียงพอละ? หรือว่าแม้ท้ายที่สุดแล้วไม่ว่าคุณจะมีสายป่านยาวแค่ไหนก็ไม่สามารถเอาชนะเกมส์การพนันได้กันแน่นะ?

\subsection{หาความจริง}

ตอนนี้เราได้โอกาสที่เราจะแพ้ โอกาสที่เราจะชนะ จำนวนเงินที่เราจะแพ้ทั้งหมดและจำนวนเงินที่เราจะได้ทั้งหมดแล้วในการเล่นด้วยวิธีการมาติงเกล (Martingale) นี้ดังนี้

โอกาสที่จะแพ้ในสมการ (2.1) โอกาสที่เราจะชนะในสมการ (2.2) จำนวนเงินที่เราอาจต้องเสียหากแพ้จากในสมการ (2.6) และจำนวนเงินที่เราจะได้ในการชนะแต่ละครั้งจากสมการ (2.5) เราจะได้ตามตามรางด้านล่างดังนี้

\begin{table}[hbt]
\centering
\begin{tabular}{|c|c|c|c|c|}
\hline
Games & ความน่าจะเป็นที่จะแพ้ & ความน่าจะเป็นที่จะชนะ & เงินที่จะเสียทั้งหมดหากแพ้ & เงินที่จะได้หากชนะ \\
\hline
ติดกัน 6 เกมส์ & $\frac{1}{64}$ & $1-\frac{1}{64}$ & $2^6-1$ & $1$ \\
ติดกัน 7 เกมส์ & $\frac{1}{128}$ & $1-\frac{1}{128}$ & $2^7-1$ & $1$ \\
ติดกัน 8 เกมส์ & $\frac{1}{256}$ & $1-\frac{1}{256}$ & $2^8-1$ & $1$ \\
\hline
\end{tabular}
\caption{ความน่าจะเป็นของเกมส์ 50-50 ที่เล่นด้วยวิธี Maringales}
\end{table} 

ถ้าเราลองเอาค่าเหล่านี้มาคำนวณด้วยคำถามที่ว่า "โอกาสที่เราชนะด้วยจำนวนเงินเหล่านี้กับโอกาสที่เราจะเสียเงินจำนวนนี้อันไหนจะมีค่ามากกว่ากัน" เราจะได้สมการดังนี้ 

กำหนดให้ $EV$ เป็นค่าของ "โอกาสที่เราชนะด้วยจำนวนเงินเหล่านี้กับโอกาสที่เราจะเสียเงินจำนวนนี้อันไหนจะมีค่ามากกว่ากัน" เราจะได้
\begin{center}
$EV=(1-(\frac{1}{2^n}))(1) - (\frac{1}{2^n})(2^n-1)$\\
(2.7)\\
$EV=(\frac{2^n-1}{2^n}) - (\frac{2^n-1}{2^n})$\\
$EV=0$\\
(2.8)
\end{center}

ดังนั้นไม่ว่าเราจะกำหนดให้ $n$ เป็นค่าใด ๆ ก็ตาม ค่า $EV$ ก็ยังมีค่าเป็น 0 เพราะท้ายที่สุดจากสมการ (2.8) ค่า $EV$ ไม่ขึ้นกับค่า $n$ ใด ๆุ เช่นไม่ว่าเราจะแทนค่า 6, 7, หรือ 8 จากตัวอย่างของเราผลก็ได้ผลออกมาว่า $EV=0$ เท่ากัน ดังตัวอย่างข้างล่าง

\begin{center}
$EV_6=(1-(\frac{1}{2^6}))(1) - (\frac{1}{2^6})(2^6-1)$\\
$EV_6=(1-(\frac{1}{64}))(1) - (\frac{1}{64})(64-1)$\\
$EV_6=\frac{63}{64} - \frac{63}{64}=0$\\
$EV_7=\frac{127}{128} - \frac{127}{128}=0$\\
$EV_8=\frac{255}{256} - \frac{255}{256}=0$\\
\end{center}

ดังนั้นจากตัวอย่างข้างต้นจะเห็นได้ว่าโดยภาพรวมแล้วผู้เล่นที่เล่นด้วยวิธีมาร์ติงเกล (Martingale) ไม่ได้มีโอกาสชนะหรือแพ้มากกว่ากันเลยในเกมส์ที่ 50-50 ที่แท้จริง จะมีก็เพียงความสนุกสนาน ภูมิใจที่มีความรู้สึกว่าโอกาสชนะมากกว่าแพ้แต่จงอย่าลืมว่า "เมื่อใดที่เกมส์แพ้คุณจะต้องเสียเงินที่เล่นได้มาทั้งหมด" ไม่ต้องเป็นการโกง กลโกง โชค เทพพรหมหรือ ฮวงจุ้ยช่วยแต่อย่างไรทุกอย่างมันเป็นกฏกติกาของคณิตศาสตร์ของธรรมชาติอยู่อย่างนี้แต่แรกอยุ่แล้ว แต่อย่างไรก็ตามถ้าเกมส์ทั้งหมดมันเป็น 50-50 จริง ๆ ความที่ผล $EV=0$  ไม่เพียงผู้เล่นที่จะไม่ชนะแต่ คาสิโนก็ไม่ได้กำไรด้วยเช่นกัน

ทว่าอย่างที่ผมกล่าวไว้ตั้งแต่ต้นบทไม่มีเกมส์ใดในคาสิโนหรือการลงทุนหรือการพนันที่เป็น 50-50 อย่างแท้จริง ในโลกของความเป็นจริงในการเล่นเกมส์อย่าง Baccarat ผลที่ออกมานั้นไม่ได้เป็น 50-50 เสียทีเดียว แต่เป็น 45.8597423 - 44.6246609 หาเรานับแต้มที่ผลเป็นเสมอด้วยแต่เมื่อเราอ้างว่าเมื่อผลเสมอไม่มีได้-เสียผลก็ยังไม่ใช่ 50-50 อยู่ดีแต่จะอยู่ที่ประมาณ 49.2614426 - 50.7385574 อาจจะดูใกล้เคียง 50-50 นั่นแหละครับแต่ผลที่ความน่าจะเป็นตรงนี้ไม่ใช่ 50-50 นี้ที่เป็นหลักประกันที่ทำให้คาสิโนสามารถชนะได้ภาษาบ้านเราเรียกว่ามีภาษีกว่า และมีเงินในการบริหารจัดการคาสิโน มีห้องเกมส์หรู ๆ ให้ผู้เล่นเข้ามาเล่นได้อย่างโอ่โถงอลังการก็เพราะความแตกต่างของความน่าจะเป็นที่ไม่ถึง 1\% นี้แหละ

ซึ่งก็สามารถแสดงออกมาได้ตามคณิตศาสตร์ได้ว่าจะมีผลต่างออกกไปอย่างไร ซึ่งจะนำเสนอต่อไป

เอาล่ะเดี๋ยวคนที่ไม่รู้จัก Baccarat จะงงว่าคุยถึงอะไร ดังนั้นเรามาพูดถึงเกมส์ Baccarat ก่อนดีกว่า 


\newpage
\section{วิธีการเล่นบาคารา (Bacccarat)}
บาคารา (Baccarat) เป็นเกมส์ที่ยอดฮิตในแถบเอเซียแต่ในแถบยุโรปกับอเมริกาจะฮิตเล่น Blackjack มากกว่า ถ้าไปคาสิโนที่มาเก๊า (Macau) ก็จะมีโต๊ะของ Baccarat มากกว่า Blackjack แต่หากได้มีโอกาสไปคาสิโนในยุโรปหรือในอเมริกเราจะพบว่ามีโต๊ะ Blackjack มากกว่าโต๊ะของ Baccarat เพราะความที่ Baccarat เป็นเกมส์ที่เล่นง่ายแต่แท้จริงแล้ว มันมีกฏกติกามารยาทของเกมส์ที่ต้องจำอยู่ไม่น้อย (สำหรับ Dealer) แต่สำหรับคนเล่นเป็นเรื่องที่ง่ายมาก วิธีการเล่นของผู้เล่นก็แค่ทายว่าฝั่ง Banker หรือ Player จะมีคะแนนที่มากกว่ากัน ถ้าทายถูกก็ชนะไป 
\begin{center}
\emph{** note **}\\
\textit{ฝั่ง Player ไม่ได้หมายถึงฝั่งที่ผู้เล่นเลือกและ ฝั่ง Banker ก็ไม่ได้หมายถึงฝั่งของ Dealer แท้จริงแล้วเป็นการตั้งชื่อเพื่อให้แบ่งเป็นสองฝั่งโดยที่ผู้เล่นสามารเลือกทายได้ทั้งฝั่ง Banker และ Player}
\end{center}

การนับคะแนนของ Baccarat เหมือนกับป๊อกเด้งของไทยเลย คือคะแนนของไพ่แต่ละใบก็คือตัวเลขที่อยู่บนหน้าไพ่เอง แต่ค่าของ J Q K มีค่าเป็น 0 และค่าของ A นับเป็นหนึ่ง หากค่าของเลขเกิน กว่า 9 เราจะนับเฉพาะค่าของตัวเลขหลักหน่วยเป็นคะแนนของฝั่งนั้น เช่น 18 นับเป็น 8, 16 นับเป็น 6, 22 นับเป็น 2 เป็นต้น ดังนั้น 19 จะชนะ 22 16  จะชนะ 21 เพราะเราดูแค่หลักหน่วยในการนับคะแนน 

การจะคำนวนโอกาสแพ้ชนะของทุก ๆ เกมส์จะขึ้นอยู่กับกฏกติกามารยาทของเกมส์ ส่วนกฏกติกาของการเปิดไพ่ Baccarat นั้นอาจจะซับซ้อนนิดหนึ่งทั้งนี้ขึ้นอยู่กับมุมมองของผู้เล่น ถ้าเล่นแค่สนุก รอ Dealer เปิดก็ไม่มีอะไรมากโอกาสความเสี่ยงอยู่ที่ใกล้เคียง 50-50 พอควรถือว่าเป็นเกมส์ที่ให้เล่นสนุกตื่นเต้นได้เกมส์หนึ่งในคาสิโน แต่ในระยะยาวคณิตศาสตร์มันจะชี้ให้เราเห็นว่า "The House Always Win" เจ้ามือชนะเสมอนั้นเป็นเรื่องจริง

ในการเปิดไพ่ Baccarat นั้นมีกฏกติกาชัดเจนว่าเริ่มต้นทั้งฝ่าย Player และ Banker จะได้รับไพ่ฝั่งละ 2 ใบ โดยฝั่ง Player จะต้องเปิดไพ่ก่อนเสมอ หลังจากนั้นฝั่ง Banker ก็จะเปิด ถ้าในขั้นตอนนี้หากมีฝ่ายใดฝ่ายหนึ่งได้คะแนน 8 หรือ 9 เกมส์ก็จะจบเท่านี้ เรามักเรียกฝั่งที่ได้ 8 หรือ 9 คะแนนว่า "ป๊อก" (Natural Hands) และจะไม่มีการหยิบไพ่เพิ่มนับคะแนนเลยซึ่งก็เป็นปกติคือฝ่ายที่ได้คะแนนมากกว่าจะเป็นฝ่ายชนะไป 

ถ้าไม่มีฝ่ายใดได้ 8 คะแนนหรือ 9 คะแนนในสองใบแรกเกมส์จะดำเนินการต่อโดยหากฝั่ง Player ได้คะแนนน้อยกว่า 6 ฝั่ง Player จะต้องหยิบไพ่เพิ่มอีก 1 ใบเป็น 3 ใบ (บังคับ) และถือคะแนนนี้เป็นที่สุด ต่อไปเป็นการหยิบไพ่เพิ่มของฝั่ง Banker แต่การหยิบไพ่ของฝั่ง Banker ขึ้นอยู่กับคะแนนเดิมของฝั่ง Banker เองและ คะแนนไพ่ที่หยิบได้ของฝั่ง Player (ไม่ใช่คะแนนรวมของฝั่ง Player) ซึ่งจะแสดงได้ดังตารางข้างล่างนี้ โดยที่ในแนวตั้งเป็นกรณีที่คะแนนรวมของฝั่ง Banker ใด ๆ และในแนวนอนจะเป็นคะแนนของไพ่ใบที่ 3 ที่ฝั่ง Player หยิบได้

\begin{table}[htb]  
%[htb] table optios to put in place with paragraph other wise it be at the top of a page
\centering
\begin{tabular}{|c|c|c|c|c|c|c|c|c|c|c|}
\hline
Player Card & 0 & 1 & 2 & 3 & 4 & 5 & 6 & 7 & 8 & 9\\
\hline  
0 & d & d & d & d & d & d & d & d & d & d\\
1 & d & d & d & d & d & d & d & d & d & d\\
2 & d & d & d & d & d & d & d & d & d & d\\
3 & d & d & d & d & d & d & d & d & s & d\\
4 & s & s & d & d & d & d & d & d & s & s\\
5 & s & s & s & s & d & d & d & d & s & s\\
6 & s & s & s & s & s & s & d & d & s & s\\
7 & s & s & s & s & s & s & s & s & s & s\\
8 & s & s & s & s & s & s & s & s & s & s\\
9 & s & s & s & s & s & s & s & s & s & s\\
\hline

\end{tabular}
\caption{ตารางแสดงการหยิบไพ่ของฝั่ง \emph{BANKER} (แนวตั้ง) โดยที่  \emph{d}: หยิบไพ่ (draw), \emph{s}: ไม่หยิบไพ่ (stand)}
\end{table}

จะเห็นได้ว่ามีความซับซ้อนในการคำนวณค่าความน่าจะเป็นอยู่ แต่ก็เป็นสิ่งที่ทำได้และมีให้ดูทั่วไปในอินเตอร์เนต (Internet) โอกาสการหยิบไพ่ของฝั่ง Banker และ Player ถูกกำหนดขึ้นมาเพื่อให้ค่าความน่าจะเป็นของผลลัพธ์เข้าใกล้ 50-50 ที่สุดแต่ไม่วายมันไม่ 50-50 เสียทีเดียวจะมีข้างหนึ่งที่จะชนะบ่อยกว่าอีกฝั่งหนึ่งเล็กน้อย ดูได้จากตารางด้านล่างนี้

\begin{table}[htb]
\centering
\begin{tabular}{|c|c|c|c|c|}
\hline
เหตุการณ์ & combinations & ความน่าจะเป็น & จ่าย & ความน่าจะเป็นกรณีไม่รวมเสมอ\\
\hline
Banker ชนะ & 2,292,252,566,437,888 & 0.458597423 & 0.95 & 0.50682483\\
Player แพ้ & 2,230,518,282,592,256 & 0.446246609 & 1 & 0.49317517\\
เสมอ & 475,627,426,473,216 & 0.095155968 & 8 & \\
\hline
\end{tabular}
\caption{แสดงความน่าจะเป็นของแต่ละเหตุการณ์ที่อาจเกิดขึ้นได้ทั้งหมด}
\end{table}

จากตารางข้างต้นจะเห็นได้ว่าเมื่อผู้เล่นทายฝั่ง Banker ชนะผู้เล่นจะได้เงินเพียง 95\% ของมูลค่าลงทุน (เงินเดิมพัน) หรือ 1 ต่อ 0.95 แต่ในทางกลับกันหากผู้เล่นเล่นฝั่ง Player ผู้เล่นจะได้รับผลตอบแทนเท่ากับจำนวนลงทุนคือ 1 ต่อ 1 แต่ถ้าคุณทายว่าจะเสมอแล้วผลลัพธ์ออกมาว่าเป็น เสมอคนจะได้ผลตอบแทนถึง 8 เท่า (ฟังดูน่าสนใจนะครับ)

คุณอาจมองว่า "มันก็ไม่ได้แย่ขนาดนั้น เราก็เล่นแค่ฝั่ง Player สิ" แต่ช้าก่อนเรามาเรียนรู้ใน Concept ของ "ค่ามุ่งหวัง" หรือ "ค่าคาดหวัง" (Expected Value) กันก่อนที่จะสรุปกันไปเอง

\newpage
\section{ค่าคาดหวัง (Expected Value)}

ค่ามุ่งหวังหรือค่าคาดหวังแท้จริงแล้วมันคือค่าของการชั่งน้ำหนักโอกาสที่จะได้รับจำนวนเงินหนึ่งหักลบกับความเสี่ยงที่เราจะสูญเสียจำนวนเงินจำนวนหนึ่ง และก็เป็นสิ่งที่เราได้ทำแล้วในบทแรกในหัวข้อระบบมาร์ติงเกลส์ (Martingales) เรื่องค่ามุ่งหวังและค่าคาดหวังนี้หากเข้าใจดีแล้วจะเป็นความรู้ที่จะเป็นพื้นฐานสำคัญให้กับนักลงทุนทุก ๆ ระดับ

ลองจินตนาการว่ากำลังเล่นเกมส์หนึ่งที่กำลังไปได้สวยชนะมาเรื่อย ๆ แต่ชนะทีละน้อย ชนะมาเรื่อย ๆ จนได้เงินเป็นกอบเป็นกำแล้วอยู่ ๆ ในเกมส์หนึ่งที่เรากำลังคิดว่าจะชนะอยู่นั้น อยู่ ๆ แพ้ซะอย่างนั้นแล้วก็แพ้เยอะเสียด้วยเยอะจนสิ่งที่เคยชนะมาทั้งหมดหายไปหมดเลย รวมทั้งมากกว่านั้นอีกด้วย ซึ่งเรื่องนี้มีจริง มีบ่อย และไม่ได้เกี่ยวกับโชคลาภ หรือการโกงแต่อย่างใด (ผมไม่ได้บอกว่าไม่มีการโกงในคาสิโน แต่จะชี้ว่าคาสิโนนั้นไม่จำเป็นต้องโกงเขาก็ชนะขาดลอยอยู่แล้วในเชิงคณิตศาสตร์) และ Concept เกี่ยวกับ ค่าคาดหวัง (Expected Value) นี้แหละที่จะชี้วัดว่าเกมส์นี้เราควรจะเล่นหรือไม่

โดยทั่ว ๆ ไปแล้วผลของค่าคาดหวังจะออกมาในสามลักษณะคือ เป็นบวก เป็นลบและเป็น 0

\begin{description}
\item[ค่าคาดหวังเป็นบวก] แสดงว่าเป็นการลงทุนที่จะให้ผลลัพธ์ที่เป็นกำไรกับผู้ลงทุน ยิ่งเป็นบวกมากยิ่งน่าลงทุนมาก
\item[ค่าคาดหวังเป็นลบ] แสดงว่าเป็นการลงทุนที่จะให้ผลลัพธ์ที่เป็นกำไรกับผู้เล่น ยิ่งเป็นลบมากยิ่งควรหนีให้ห่าง
\item[ค่าคาดหวังเป็นศูนย์]  แสดงว่าการลงทุนนี้โดยรวมแล้วไม่เป็นผลกำไรให้กับผุ้เล่นคนใด
\end{description}

ตอนนี้เรารู้แล้วว่าผลลัพธ์ของค่าคาดหวังหรือค่ามุ่งหวังนั้นมีความหมายอย่างไร ตอนนี้สิ่งที่สำคัญหว่านั้นคือเราต้องรู้วิธีหาค่ามุ่งหวังหรือค่าคาดหวังเหล่านี้ให้ได้เพื่อจะได้รู้ว่าการลงทุนนี้เป็นการลงทุนที่คุ้มค่ากับการลงทุนหรือไม่ 

เรามาลองคิดดูว่าตอนนั้นเราทำยังไงนะตอนที่เราคำนวณหาค่ามุ่งหวังของวิธีการเล่นแบบมาร์ติงเกลส์ (Martingale's) เราได้หาความน่าจะเป็นที่เราจะชนะ (โอกาส - C) ความน่าจะเป็นที่เราจะแพ้ (ความเสี่ยง - R) มูลค่าเงินที่จะได้หากเราชนะ (Win - W) และมูลค่าเงินที่เราจะเสีย (Loss - L) แล้วเราเอาโอกาสที่ชนะคูณกับจำนวนเงินที่เราจะชนะ ลบด้วยความเสี่ยงลบด้วยมูลค่าเงินที่เราจะเสีย และสิ่งที่เราจะได้นี้แหละเรียกว่า "ค่าุคาดหวัง" (Expected Value - $E_v$) เราสามารถเขียนเป็นสมการได้ดังนี้

\begin{center}
$E_v = CW - LR$\\
(4.1)
\end{center} 

ถ้าเราเรียกโอกาสที่จะได้รับมูลค่าหนึ่งว่า "ค่าคาดหวังที่จะชนะ" (Expected Win)  เขียนเป็นตัวแปรว่า $E_W$ และเรียกความเสี่ยงที่จะสูญเสียมูลค่าลงทุนว่า "ค่าคาดหวังว่าสูญเสีย" (Expected Loss) เขียนด้วยตัวแปรว่า $E_L$ เราจะได้ว่า

 \begin{center}
 เมื่อ\\
$E_W=CW$\\
$E_L=LR$\\
เราจะได้\\
$E_v = E_W - E_L$\\
(4.2)
\end{center} 

ซึ่งหากโอกาสและมูลค่าที่ชนะมีมากกว่าหนึ่งแบบ และความเสี่ยงที่เราจะสูญเสียเงินมากกว่าหนึ่งแบบเราต้องรวมค่าเหล่านั้นไว้แล้วทำการเอามาลบกันทีเดียว เราจะได้สมการใหม่ของค่าคาดหวัง (Expected Value) ว่าเป็นความต่างระหว่างผลรวมของค่าคาดหวังที่จะชนะและผลรวมของค่าคาดหวังว่าสูญเสียว่า หากจำนวนพจน์ของ $E_W$ มี $i$ และ $E_L$ มี $j$ แล้วเราจะเขียนสมการได้ว่า

\begin{center}
$$E_v=\sum_{i=1}^{i}E_{W_i}-\sum_{j=1}^{j}E_{L_j}$$\\(4.3)\\
\end{center}

และในที่สุดหากเราไม่แบ่งเลยว่าจะได้หรือจะเสียเราคิดเสียว่าเป็นโอกาสและมูลค่าเท่านั้น หากเราให้มูลค่าที่จะได้มีมูลค่าเป็น + และมูลค่าที่จะเสียมีค่าเป็น - เราจะได้ว่าค่าคาดหวังคือผลรวมของของค่าคาดหวังของเหตุการณ์ต่าง ๆ เท่านั้น หากเราให้ $ev$ เป็นค่าคาดหวังในแต่ละเหตุการณ์ย่อยเหล่านี้ $p$ เป็นความน่าจะเป็น (Probability) ของแต่ละเหตุการณ์ย่อย และให้ $v$ เป็นมูลค่าที่เกี่ยวเนื้องด้วยความน่าจะเป็นนั้น จะเขียนเป็นสมการได้ว่า

\begin{center}
เมื่อ\\
$ev=pv$\\
$$E_v=\sum_{i=1}^{i}ev_i=p_1v_1+p_2v_2+p_3v_3...+p_iv_i$$\\(4.4)\\
\end{center}

หากเรานำความน่าจะเป็นและมูลค่าที่จะได้เสียในตารางผลลัพธ์ของการเล่นบาคารา (Baccarat) ในบทที่แล้วมาคำนวณ ถ้าให้ $p_B$ เป็นความน่าจะเป็นที่ Banker ชนะ และมูลค่าที่จะได้รับ $v_B$ และให้ $p_P$ เป็นความน่าจะเป็นที่ Player ชนะและมูลค่าที่จะได้รับ $v_P$ หากมูลค่าลงทุนทั้งหมดเท่ากันคือ 1 หน่วย เราเอาเฉพาะกรณีที่ไม่นับการเสมอเราจะได้ว่า 

\begin{center}
ถ้าให้ $E_{v_B}$ เป็นการลงทุนในฝั่ง Banker โดยที่\\
$p_B=0.50682483$, $v_B=0.95$\\
เราจะได้ว่า\\
$E_{v_B}=p_B*v_B+p_P*v_P$\\
ทำการแทนค่า จะได้\\
$E_{v_B}=0.50682483*0.95+0.49317517*(-1)$\\
$E_{v_B}=0.481483588-0.49317517$\\
$E_{v_B}=-0.011691582$\\
\end{center}

ดังนั้นค่าคาดหวังในการลงทุนฝั่ง Banker มีค่า -0.011691582 ในทุก ๆ การลงทุน 1 หนึ่งหน่วย หรือเทียบได้ว่า -1.17\% ดังนั้นจากคำนิยามของค่าคาดหวังจึงกล่าวได้ว่าการลงทุนในฝั่ง Banker นี้ไม่เป็นการลงทุนที่น่าลงทุนเลย 


\begin{center}
ถ้าให้ $E_{v_P}$ เป็นการลงทุนในฝั่ง Player โดยที่\\
$p_P=0.49317517$, $v_P=1$\\
เราจะได้ว่า\\
$E_{v_P}=p_B*v_B+p_P*v_P$\\
ทำการแทนค่า จะได้\\
$E_{v_P}=0.50682483*(-1)+0.49317517*1$\\
$E_{v_P}=0.49317517-0.50682483$\\
$E_{v_P}=-0.01364966$\\
\end{center}

ดังนั้นค่าคาดหวังในการลงทุนฝั่ง Player มีค่า -0.01364966 ในทุก ๆ การลงทุน 1 หนึ่งหน่วย หรือเทียบได้ว่า -1.36\% ดังนั้นจากคำนิยามของค่าคาดหวังจึงกล่าวได้ว่าการลงทุนในฝั่ง Player นี้ไม่เป็นการลงทุนที่น่าลงทุนเลย 

จากข้างบนเราจะเห็นว่าค่าคาดหวังของการลงทุนทั้งในฝั่ง Banker และ Player ต่างก็มีค่าเป็นลบนั่นเป็นเพราะเดิมในฝั่ง Player นั่นโอกาสที่จะชนะนั้นน้อยกว่า 50\% อยู่แล้ว แต่เมื่อผลตอบแทนและมูลค่าลงทุนที่เท่ากันจึงมีผลลัพธ์เป็นลบ แต่ในฝั่ง Banker นั้นมีโอกาสที่จะออกเป็น Banker มากกว่าอยู่แล้วถ้ามีผลตอบแทนและมูลค่าลงทุนเท่ากันย่อมให้ผลค่าคาดหวังเป็นบวก แต่เพราะฝั่ง Banker จะให้ผลตอบแทนเพียง 95\% ของเงินลงทุนเมื่อเอามาชั่งน้ำหนักกันดูกับความน่าจะเป็นแล้วจึงทำให้ผลค่าคาดหวังของการลงทุนในฝั่ง Banker ก็เป็นลบด้วย

ดังนั้นจากการคำนวณค่าคาดหวังของการเล่น Baccarat ข้างต้นทำให้คุณเห็นว่า ในการเล่น Baccarat ไม่ว่าคุณจะเล่นฝั่งไหนคุณก็มีแต่จะเสีย อ่าวและถ้าเราทายว่าเสมอ (Tie) ล่ะ เอาล่ะเรามาลองคำนวณกันดู จากสูตร (4.4) ถ้าให้ $p_T$ เป็นความน่าจะเป็นที่จะออกเสมอ และ $v_T$ เป็นผลตอบแทนของค่าน่าจะเป็นนี้

\begin{center}
ถ้าให้ $E_{v_T}$ เป็นการลงทุนว่าเกมส์จะเสมอ (Tie) โดยที่\\
$p_T=0.095155968$, $v_T=8$\\
เราจะได้ว่า\\
$E_{v_T}=p_T*v_T+p_P*v_P+p_B*v_B$\\
ทำการแทนค่า จะได้\\
$E_{v_T}=0.095155968*8+0.446246609*(-1)+0.458597423*(-1)$\\
$E_{v_T}=0.761247744+(-0.446246609)+(-0.458597423)$\\
$E_{v_T}=-0.143596288$\\
\end{center}

ดังนั้นค่าคาดหวังในการลงทุนทายว่าเสมอ มีค่าติดลบถึง -0.143596288 ในทุก ๆ การลงทุน 1 หนึ่งหน่วย หรือเทียบได้ว่า -14.36\% ดังนั้นจากคำนิยามของค่าคาดหวังจึงกล่าวได้ว่าการลงทุนว่าเสมอ นี้ก็ไม่เป็นการลงทุนที่น่าลงทุนเลย และการลงทุนในการเล่นแบบ Tie ถือเป็นการลงทุนที่ไม่คุ้มที่สุดในการเล่น Baccarat 

หากเราไปเล่นด้วยความต้องการเอาสนุกสนาน เอาความตื่นเต้น Baccarat ถือเป็นเกมส์หนึ่งที่มีค่าเล่นถูกที่สุดคือประมาณ -1\%  แต่การลงทุนว่าจะเสมอจะเป็นความตื่นเต้นที่แพงที่สุดในเกมส์คือ -14\%

\subsection{ค่าคาดหวังของมาร์ติงเกล (Martingale's Expected Value)}
จากสมการที่ (2.7) เราจะเห็นว่าเราจะหาค่าคาดหวังของ Martingale's ได้จาก 1 ลบด้วยค่าความน่าจะเป็นที่เราจะเสีย $p$ ยกกำลัง $n$ โดยที่ $n$ คือจำนวนเกมส์ที่จะยอมเล่นแพ้ติด ๆ กัน คูณกับ 1 ทั้งหมดนี้เป็น $E_W$ (Expected Win) และเราสามารถหา Expected Loss ได้จาก ความน่าจะเป็นที่เราจะเสีย $p$ ยกกำลัง $n$ คูณด้วย $(2^n-1)$ สามารถเขียนออกมาเป็นสมการได้ว่า

\begin{center}
$EV=E_W-E_L$\\
$EV=(1-p^n)*1 - p^n(2^n-1)$\\
(4.5)\\
\end{center}

เมื่อพิจารณา (4.5) เราจะเห็นว่าจริง ๆ แล้ว 1 ตัวแรกที่เป็นตัวคูณคือค่าตอบแทนหากสามารถชนะได้ในครั้งที่ $n$ แต่ในกรณีนี้เห็นเป็น 1 เพราะ $1^n=1$ ดังนั้นหากเราให้ $w$ เป็นค่าตอบแทนในการที่ชนะครั้งหนึ่งเราจะได้ว่า

\begin{center}
$EV=E_W-E_L$\\
$EV=(1-p^n)*w^n - p^n(2^n-1)$\\
(4.6)\\
\end{center}

ซึ่งหากนำ (4.6) มาแทนค่าในตารางที่ 3 การลงทุนด้วยวิธีการมาร์ติงเกลในฝั่ง Banker เราจะได้

\begin{center}
$EV=(1-p^n)*w^n - p^n(2^n-1)$\\
$EV_B=(1-p_P^n)*w^n - p_P^n(2^n-1)$\\
$EV_B=(1-0.49317517^n)*0.95^n - 0.49317517^n(2^n-1)$\\
(4.7)\\
\end{center}

นำ (4.6) มาแทนค่าในตารางที่ 3 การลงทุนด้วยวิธีการมาร์ติงเกลในฝั่ง Player เราจะได้

\begin{center}
$EV=(1-p^n)*w^n - p^n(2^n-1)$\\
$EV_P=(1-p_B^n)*w^n - p_B^n(2^n-1)$\\
$EV_P=(1-0.50682483^n)*1 - 0.50682483^n(2^n-1)$\\
(4.8)
\end{center}

นำ (4.6) มาแทนค่าในตารางที่ 3 การลงทุนทายว่าเสมอใน Baccarat ด้วยวิธีการมาร์ติงเกล เราจะได้

\begin{center}
$EV=(1-p^n)*w^n - p^n(2^n-1)$\\
$EV_T=(1-p_P^n)*w^n - p_P^n(2^n-1)$\\
$EV_T=(1-0.904844032^n)*8^n - 0.904844032^n(2^n-1)$\\
(4.9)
\end{center}

\newpage
\section{ระบบกลยุทธการลงทุน}

\newpage
\section{เกมส์การพนันอื่น ๆ}

\newpage
\section{ความเข้าใจผิดของนักพนัน Gambler's Falacies}

\newpage
\section{สรุป}

\newpage
\section{อ้างอิง}





•
\end{document}